%===============================================================================
% Zweck: KTR-Meta-Vorlage
%===============================================================================
\input{\meta/config/language}
\newif\ifgit
\newif\ifseminar
\newif\ifpresentation
% Input files from meta package
\IfFileExists{config/metainfo}{\input{config/metainfo}}{\gitfalse}
\usepackage[utf8]{inputenc}
\usepackage{fancyhdr}
\usepackage[T1]{fontenc}
\ifgit
  \usepackage[mark]{gitinfo2}
\fi
\usepackage{ae}
\usepackage{color}
\usepackage{amsmath}
\usepackage{amsfonts}

%%   Fuer anspruchsvolle Tabellen   %%
\usepackage{longtable, colortbl}
\usepackage{multicol, multirow}

\usepackage[pdftitle={\@title},pdfauthor={\@author},pdftex,bookmarksopen,bookmarksnumbered]{hyperref}
\usepackage[pdftex]{graphicx}
\usepackage{float}
\usepackage{tikz}
\usepackage{pgfplots}
\usetikzlibrary{arrows,shapes,fit,positioning,decorations,backgrounds,shadows}

\pdfcompresslevel=9

% Code-Hervorhebung
% Quellcode
\usepackage{verbatim}            % Quellcode einbinden (\verbatiminput) standardpaket
\usepackage{moreverb}
% PseudoCode
\usepackage{algorithm}
\usepackage{algpseudocode}
%\usepackage{algorithmicx}

\floatname{algorithm}{\algo}
\algrenewcommand{\algorithmiccomment}[1]{\hskip1em\textcolor{gray!60}{$\rhd$ #1}}
\renewcommand{\listalgorithmname}{\loa}
\def\algorithmautorefname{\algo}


%%   intoc zur Aufnhame des Abkuerzungs- und Symbolverzeichnisses ins Inhaltsverzeichnis
\usepackage[intoc]{nomencl}
\setlength{\nomlabelwidth}{.20\hsize}
\renewcommand{\nomlabel}[1]{#1 \dotfill}
\setlength{\nomitemsep}{-\parsep}
\makenomenclature

\renewcommand{\nomname}{\abbr}

%%   Hervorhebung der Abkuerzungsbuchstaben   %%
\usepackage[normalem]{ulem}
\newcommand{\m}[1]{\uline{#1}}

% ausf\"{u}hrlichere Fehlermeldungen
\errorcontextlines=999
%
% Page-Layout: A4 aus Header
% Alternative
\setlength\headheight{14pt}
\setlength\topmargin{-15,4mm}
\setlength\oddsidemargin{-0,4mm}
\setlength\evensidemargin{-0,4mm}
\setlength\textwidth{160mm}
\setlength\textheight{252mm}
%
%% Absatzeinstellungen
\setlength\parindent{0mm}
\setlength\parskip{2ex}

\input{\meta/config/unibaColors}
\ifgit
  \newcommand{\gitkeys}{\gitAbbrevHash, \gitAuthorIsoDate, \gitAuthorName }
\fi

\ifpresentation
\usetheme{UniBa\ratio}
%\usefonttheme{
%	default | professionalfonts | serif |
%	structurebold | structureitalicserif |
%	structuresmallcapsserif
%}
\usefonttheme{professionalfonts}
%\useinnertheme{
%	circles | default | inmargin |
%	rectangles | rounded
%}
\useinnertheme{rectangles}
%\useoutertheme{
%	default | infolines | miniframes |
%	shadow | sidebar | smoothbars |
%	smoothtree | split | tree
%}
%\useoutertheme{split}
\setbeamercovered{transparent}

% Without navigation symbols
\beamertemplatenavigationsymbolsempty
\fi

\makeatletter
\hypersetup{pdftitle={\@title}, pdfauthor={\@author}, linktoc=page, pdfborder={0 0 0 [3 3]}, breaklinks=true, linkbordercolor=unibablueI, menubordercolor=unibablueI, urlbordercolor=unibablueI, citebordercolor=unibablueI, filebordercolor=unibablueI}
%% Define a new 'leo' style for the package that will use a smaller font.
\def\url@leostyle{%
  \@ifundefined{selectfont}{\def\UrlFont{\sf}}{\def\UrlFont{\small\ttfamily}}}
\makeatother
%% Now actually use the newly defined style.
\urlstyle{leo}

\graphicspath{{images/}{\meta/config/images/}{\meta/images}}

\ifpresentation
\else
\makeatletter
\renewcommand{\maketitle} {
  \begin{titlepage}
  \centering
    \begin{minipage}[t]{16cm}
      \hfill
      \begin{minipage}{12cm}
            \centering
        \uni
        \\[12pt]%
        {\Large \chair\\[.5em]%
        \large \chairsub}%
      \end{minipage}
      \hfill
      \begin{minipage}{3cm}
        \includegraphics[height=28mm]{\meta/config/images/logo} %height=26mm
      \end{minipage}
    \end{minipage}\\[70pt]%[50pt]
    {\Large\bf \ifseminar\seminar\else\project\fi}\\[36pt]
    {\LARGE \@title}\\[80pt]
    \ifseminar%
    {\Large\bf \topic:}\\[36pt]
    {\LARGE\bf \subtitle}\\
    \fi%
    \vfill
    \begin{minipage}{\textwidth}
      \center
      \submitter:\\
      {\Large \@author \\[18pt]}
      \lsupervisor: \supervisor \\[12pt]
      Bamberg, \@date\\
      \semester
    \end{minipage}
  \end{titlepage}
}
\makeatother
\fi

\ifpresentation
\renewcommand{\gitMarkFormat}{\color{unibagrayI}\tiny\sffamily}
\else
\renewcommand{\gitMarkFormat}{\color{unibagrayI}\small\sffamily}
\fi

\numberwithin{equation}{section}
%
%===============================================================================
% zentrale Layout-Angaben und Befehle
%===============================================================================
%
%#1 Breite
%#2 Datei (liegt im image Verzeichnis)
%#3 Beschriftung
%#4 Label fuer Referenzierung
\newcommand{\image}[4]{%
\begin{figure}[H]%
\centering%
\includegraphics[width=#1]{#2}%
\caption{#3}%
\label{#4}%
\end{figure}%
}

%#1 Datei (liegt im graphic Verzeichnis)
%#2 Beschriftung
%#3 Label fuer Referenzierung
%#4 Skalierungsfaktor
\newcommand{\scaletikzimage}[4]{%
\begin{figure}[H]%
\centering%
\scalebox{#4}{%
\IfFileExists{graphic/#1.tikz}{\input{graphic/#1.tikz}}{
\IfFileExists{\meta/exampleGraphic/#1.tikz}{\input{\meta/exampleGraphic/#1.tikz}}{%
\colorbox{red}{Put your tikz file in the \texttt{graphic} folder}%
}}}%
\caption{#2}%
\label{#3}%
\end{figure}
}

%#1 algorithm name
%#2 algorithm label
%#3 file name in code-folder
\newcommand{\pseudo}[3]{%
\small%
\begin{algorithm}[H]%
\caption{#1}%
\label{#2}%
\IfFileExists{code/#3.tex}{\input{code/#3.tex}}{%
\IfFileExists{\meta/exampleCode/#3.tex}{\input{\meta/exampleCode/#3.tex}}{%
\colorbox{red}{Put your code file in the \texttt{code} folder}%
}}%
\end{algorithm}%
\normalsize%
}

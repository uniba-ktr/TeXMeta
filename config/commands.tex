%===============================================================================
% Zweck: KTR-Meta-Vorlage
%===============================================================================
\input{\meta/config/language}
\newif\ifgit
\newif\ifseminar
\newif\ifpresentation
\newif\ifposter
\newif\ifthesis
\newif\iftodo
% Input files from meta package
\IfFileExists{config/metainfo}{\input{config/metainfo}}{\gitfalse}
\usepackage[utf8]{inputenc}
\usepackage{fancyhdr}
\usepackage[T1]{fontenc}
\ifgit
  \usepackage[mark]{gitinfo2}
\fi
\usepackage{ae}
\usepackage{color}
\usepackage{amsmath}
\usepackage{amsfonts}

%%   Fuer anspruchsvolle Tabellen   %%
\usepackage{longtable, colortbl}
\usepackage{multicol, multirow}

\usepackage[pdftitle={\@title},pdfauthor={\@author},pdftex,bookmarksopen,bookmarksnumbered]{hyperref}
\usepackage[pdftex]{graphicx}
\usepackage{float}
\usepackage{tikz}
\usepackage{pgfplots}
\usetikzlibrary{arrows,shapes,fit,positioning,decorations,backgrounds,shadows}

\pdfcompresslevel=9

% Code-Hervorhebung
% Quellcode
\usepackage{verbatim}            % Quellcode einbinden (\verbatiminput) standardpaket
\usepackage{moreverb}
% PseudoCode
\usepackage{algorithm}
\usepackage{algpseudocode}
%\usepackage{algorithmicx}

\floatname{algorithm}{\algo}
\algrenewcommand{\algorithmiccomment}[1]{\hskip1em\textcolor{gray!60}{$\rhd$ #1}}
\renewcommand{\listalgorithmname}{\loa}
\def\algorithmautorefname{\algo}


%%   intoc zur Aufnhame des Abkuerzungs- und Symbolverzeichnisses ins Inhaltsverzeichnis
\usepackage[intoc]{nomencl}
\setlength{\nomlabelwidth}{.20\hsize}
\renewcommand{\nomlabel}[1]{#1 \dotfill}
\setlength{\nomitemsep}{-\parsep}
\makenomenclature

\renewcommand{\nomname}{\abbr}

%%   Hervorhebung der Abkuerzungsbuchstaben   %%
\usepackage[normalem]{ulem}
\newcommand{\m}[1]{\uline{#1}}

% ausf\"{u}hrlichere Fehlermeldungen
\errorcontextlines=999
%
% Page-Layout: A4 aus Header
% Alternative
\setlength\headheight{14pt}
\setlength\topmargin{-15,4mm}
\setlength\oddsidemargin{-0,4mm}
\setlength\evensidemargin{-0,4mm}
\setlength\textwidth{160mm}
\setlength\textheight{252mm}
%
%% Absatzeinstellungen
\setlength\parindent{0mm}
\setlength\parskip{2ex}

\input{\meta/config/unibaColors}
\ifgit
  \newcommand{\gitkeys}{\gitAbbrevHash, \gitAuthorIsoDate, \gitAuthorName }
\fi

\ifpresentation
\usetheme{UniBa\ratio}
%\usefonttheme{
%	default | professionalfonts | serif |
%	structurebold | structureitalicserif |
%	structuresmallcapsserif
%}
\usefonttheme{professionalfonts}
%\useinnertheme{
%	circles | default | inmargin |
%	rectangles | rounded
%}
\useinnertheme{rectangles}
%\useoutertheme{
%	default | infolines | miniframes |
%	shadow | sidebar | smoothbars |
%	smoothtree | split | tree
%}
%\useoutertheme{split}
\setbeamercovered{transparent}

% Without navigation symbols
\beamertemplatenavigationsymbolsempty
\fi

\makeatletter
\ifposter
\else
\hypersetup{pdftitle={\@title}, pdfauthor={\@author}, linktoc=page, pdfborder={0 0 0 [3 3]}, breaklinks=true, linkbordercolor=unibablueI, menubordercolor=unibablueI, urlbordercolor=unibablueI, citebordercolor=unibablueI, filebordercolor=unibablueI}
\fi
%% Define a new 'leo' style for the package that will use a smaller font.
\def\url@leostyle{%
  \@ifundefined{selectfont}{\def\UrlFont{\sf}}{\def\UrlFont{\small\ttfamily}}}
\makeatother
%% Now actually use the newly defined style.
\urlstyle{leo}


\graphicspath{{images/},{\meta/config/images/},{\meta/images/}}
\pgfplotsset{compat=1.9}

\ifpresentation
\else
\makeatletter
\renewcommand{\maketitle} {
\begin{titlepage}
\ifthesis
\ThisCenterWallPaper{1}{\meta/config/images/titlepage.pdf}

\setstretch{1.2}
\vspace*{55mm}
\begin{minipage}[t]{2cm}
\textsc{Thema:}
\end{minipage}
\begin{minipage}[t]{12cm}
\textbf{\Large \@title}\\[10mm]
\textbf{\large \@subtitle \normalsize}
\end{minipage}\\[25mm]
\centering
\Huge \textbf{\degree arbeit}\\
\vspace{1cm}
\Large
im Studiengang \studycourse\ der Fakultät Wirtschaftsinformatik und Angewandte Informatik der Otto-Friedrich-Universität Bamberg\\
\normalsize
\vfill
\begin{flushleft}
\begin{tabbing}
xxxxxxxxxxxxxxx\=xxxxxxxxxxxxxx\kill
Verfasser: \> \@author\\
Themensteller:\> \advisor \\
Abgabedatum:\> \@date\\
\end{tabbing}
\end{flushleft}
\else
  \centering
    \begin{minipage}[t]{16cm}
      \hfill
      \begin{minipage}{12cm}
            \centering
        \uni
        \\[12pt]%
        {\Large \chair\\[.5em]%
        \large \chairsub}%
      \end{minipage}
      \hfill
      \begin{minipage}{3cm}
        \includegraphics[height=28mm]{\meta/config/images/logo} %height=26mm
      \end{minipage}
    \end{minipage}\\[70pt]%[50pt]
    {\Large\bf \ifseminar\seminar\else\project\fi}\\[36pt]
    {\LARGE \@title}\\[80pt]
    \ifseminar%
    {\Large\bf \topic:}\\[36pt]
    {\LARGE\bf \subtitle}\\
    \fi%
    \vfill
    \begin{minipage}{\textwidth}
      \center
      \submitter:\\
      {\Large \@author \\[18pt]}
      \lsupervisor: \supervisor \\[12pt]
      Bamberg, \@date\\
      \semester
    \end{minipage}
\fi
\end{titlepage}
}
\makeatother
\fi

\ifgit
  \renewcommand{\gitMarkFormat}{\color{unibagrayI}\ifpresentation\tiny\else\small\fi\sffamily}
\fi

\ifthesis
% Schönere Kapitel?
\renewcommand*{\chapterformat}{%
  \thechapter\enskip
  \textcolor{gray!50}{\rule[-\dp\strutbox]{2pt}{\baselineskip}}\enskip
}
\renewcommand{\headfont}{\normalfont\sffamily\itshape}    % Kolumnentitel serifenlos
\renewcommand{\pnumfont}{\normalfont\sffamily}    % Seitennummern serifenlos
\pagestyle{scrheadings}
%\pagestyle{scrplain}
\ihead[]{\headmark}              % Kolumnentitel immer oben innen
\chead[]{}                       % Mitte leer lassen
\ohead[\pagemark]{\pagemark}     % Seitennummern immer oben aussen
%\ohead[]{}
\ofoot[]{}                       % Seitennummern in der Fusszeile loeschen
\cfoot[]{\ifgit \gitMarkFormat{\gitMarkPref\,\textbullet{}\,Branch: \gitBranch\,@\,\gitAbbrevHash{} \textbullet{} Release:\gitReln{} (\gitAuthorDate)}\fi}                       % Seitennummern in der Fusszeile loeschen
\fi

\numberwithin{equation}{section}
%
%===============================================================================
% zentrale Layout-Angaben und Befehle
%===============================================================================
%
%#1 Breite
%#2 Datei (liegt im image Verzeichnis)
%#3 Beschriftung
%#4 Label fuer Referenzierung
\newcommand{\image}[4]{%
\begin{figure}[H]%
\centering%
\includegraphics[width=#1]{#2}%
\caption{#3}%
\label{#4}%
\end{figure}%
}

%#1 Breite
%#2 Datei (liegt im image Verzeichnis)
%#3 Beschriftung
%#4 Label fuer Referenzierung
\newcommand{\pic}[2]{
\begin{figure}[H]
\centering
\includegraphics[width=#1]{#2}
\end{figure}
}


%#1 Datei (liegt im graphic Verzeichnis)
%#2 Beschriftung
%#3 Label fuer Referenzierung
%#4 Skalierungsfaktor
\newcommand{\scaletikzimage}[4]{%
\begin{figure}[H]%
\centering%
\scalebox{#4}{%
\IfFileExists{graphic/#1.tikz}{\input{graphic/#1.tikz}}{
\IfFileExists{\meta/exampleGraphic/#1.tikz}{\input{\meta/exampleGraphic/#1.tikz}}{%
\colorbox{red}{Put your tikz file in the \texttt{graphic} folder}%
}}}%
\caption{#2}%
\label{#3}%
\end{figure}
}

% You must include \usepackage[font=footnotesize]{subfig} to use this command
% #1 relative width of both figures at most 0.5
% #2 picture one in /taskXX/P1
% #3 caption of figure 1
% #4 label of figure 1
% #5 picture two in /taskXX/P2
% #6 caption of figure 2
% #7 label of figure 2
% #8 overall caption
% #9 overall label
\newcommand{\twofigures}[9]{%
  \begin{figure}[H]%
    \centerline{%
      \subfloat[#3]{%
        \includegraphics[width=#1\textwidth]{#2}%
        \label{#4}%
      }%
      \hfil%
      \subfloat[#6]{%
        \includegraphics[width=#1\textwidth]{#5}%
        \label{#7}%
      }%
    }%
    \caption{#8}%
    \label{#9}%
  \end{figure}%
}

%#1 algorithm name
%#2 algorithm label
%#3 file name in code-folder
\newcommand{\pseudo}[3]{%
\small%
\begin{algorithm}[H]%
\caption{#1}%
\label{#2}%
\IfFileExists{code/#3.tex}{\input{code/#3.tex}}{%
\IfFileExists{\meta/exampleCode/#3.tex}{\input{\meta/exampleCode/#3.tex}}{%
\colorbox{red}{Put your code file in the \texttt{code} folder}%
}}%
\end{algorithm}%
\normalsize%
}

\newcounter{saveenumi}
\newcommand{\seti}{\setcounter{saveenumi}{\value{enumi}}}
\newcommand{\conti}{\setcounter{enumi}{\value{saveenumi}}}

\ifpresentation
\resetcounteronoverlays{saveenumi}
\fi

%\ifthesis
\makeatletter
\newcommand{\erklaerung}{
\newpage
%-----------------------------------------------------------------------
\section*{Erklärung zur Beachtung der Vorschriften der Allgemeinen Prüfungsordnung (APO) WIAI}
%-----------------------------------------------------------------------
Ich erkläre hiermit gemäß  APO \S  9, Absatz 12, dass ich die vorstehende \ifthesis\degree arbeit \else Hausarbeit \fi selbständig verfasst habe und ich keine anderen als die angegebenen Quellen und Hilfsmittel benutzt habe.
\\
Des Weiteren erkläre ich, dass die digitale Fassung der gedruckten Ausfertigung der \ifthesis\degree arbeit \else Hausarbeit \fi ausnahmslos in Inhalt und Wortlaut entspricht und zur Kenntnis genommen wurde, dass diese digitale Fassung einer durch Software unterstützten, anonymisierten Prüfung auf Plagiate unterzogen werden kann.
%-----------------------------------------------------------------------
\\
%-----------------------------------------------------------------------
\section*{Declaration on the Compliance With Regulations of ``Allgemeine Prüfungsordnung'' (APO) WIAI}
%-----------------------------------------------------------------------
I hereby declare in accordance with  APO \S 9, subsection 12, that I have written the preceding \ifthesis\expandafter\MakeLowercase\expandafter{\degree}'s thesis \else report \fi independently and that I have not used any other sources or aids than those indicated above.
\\
Furthermore, I declare that the digital version of the printed copy of the \ifthesis\expandafter\MakeLowercase\expandafter{\degree}'s thesis \else report \fi corresponds without any exception to the content and wording of the printed version of the \ifthesis\expandafter\MakeLowercase\expandafter{\degree}'s thesis\else report\fi. Furthermore, I declare  that I have taken note of the regulation that this digital version of the \ifthesis\expandafter\MakeLowercase\expandafter{\degree}'s thesis \else report \fi can be the subject of an anonymous plagiarism checking supported by software.
\\[2cm]
%-----------------------------------------------------------------------

\underline{\hspace{8cm}}
\\
\mbox{Ort und Datum / Place and date}
\\[1cm]


\underline{\hspace{10cm}}
\\
\mbox{Unterschrift mit Vornamen und Familienname}
 \\
\mbox{Author's signature with first names and family name}
\newpage
}
\makeatother
%\fi

%===============================================================================
% Listing Styles
%===============================================================================
\lstset{basicstyle=\ttfamily,showstringspaces=false,commentstyle=\color{unibagrayI},keywordstyle=\color{unibablueI},breaklines=true}
\DeclareFixedFont{\ttb}{T1}{txtt}{bx}{n}{9} % for bold
\DeclareFixedFont{\ttm}{T1}{txtt}{m}{n}{9}  % for normal
\lstset{
language=Python,
basicstyle=\small,
otherkeywords={self},             % Add keywords here
keywordstyle=\small\bf\color{unibablueI},
emph={MyClass,__init__},          % Custom highlighting
emphstyle=\small\bf\color{nounibaredII},    % Custom highlighting style
stringstyle=\small\color{nounibagreenII},
commentstyle=\small\color{unibagrayI},          % Any extra options here
showstringspaces=false            %
}

\newcommand\YAMLcolonstyle{\color{nounibaredII}\mdseries}
\newcommand\YAMLkeystyle{\color{black}\bfseries}
\newcommand\YAMLvaluestyle{\color{nounibagreenII}\mdseries}

\makeatletter

% here is a macro expanding to the name of the language
% (handy if you decide to change it further down the road)
\newcommand\language@yaml{yaml}

\expandafter\expandafter\expandafter\lstdefinelanguage
\expandafter{\language@yaml}
{
  keywords={true,false,null,y,n},
  keywordstyle=\color{darkgray}\bfseries,
  basicstyle=\YAMLkeystyle,                                 % assuming a key comes first
  sensitive=false,
  comment=[l]{\#},
  morecomment=[s]{/*}{*/},
  commentstyle=\color{purple}\ttfamily,
  stringstyle=\YAMLvaluestyle\ttfamily,
  moredelim=[l][\color{orange}]{\&},
  moredelim=[l][\color{magenta}]{*},
  moredelim=**[il][\YAMLcolonstyle{:}\YAMLvaluestyle]{:},   % switch to value style at :
  morestring=[b]',
  morestring=[b]",
  literate =    {---}{{\ProcessThreeDashes}}3
                {>}{{\textcolor{red}\textgreater}}1
                {|}{{\textcolor{red}\textbar}}1
                {\ -\ }{{\mdseries\ -\ }}3,
}

% switch to key style at EOL
\lst@AddToHook{EveryLine}{\ifx\lst@language\language@yaml\YAMLkeystyle\fi}
\makeatother

\newcommand\ProcessThreeDashes{\llap{\color{cyan}\mdseries-{-}-}}

% Tikz grid
\makeatletter
\def\grd@save@target#1{%
\def\grd@target{#1}}
\def\grd@save@start#1{%
\def\grd@start{#1}}
\tikzset{
grid with coordinates/.style={
to path={%
\pgfextra{%
\edef\grd@@target{(\tikztotarget)}%
\tikz@scan@one@point\grd@save@target\grd@@target\relax
\edef\grd@@start{(\tikztostart)}%
\tikz@scan@one@point\grd@save@start\grd@@start\relax
\draw[minor help lines] (\tikztostart) grid (\tikztotarget);
\draw[middle help lines] (\tikztostart) grid (\tikztotarget);
\draw[major help lines] (\tikztostart) grid (\tikztotarget);
\grd@start
\pgfmathsetmacro{\grd@xa}{\the\pgf@x/1cm}
\pgfmathsetmacro{\grd@ya}{\the\pgf@y/1cm}
\grd@target
\pgfmathsetmacro{\grd@xb}{\the\pgf@x/1cm}
\pgfmathsetmacro{\grd@yb}{\the\pgf@y/1cm}
\pgfmathsetmacro{\grd@xc}{\grd@xa + \pgfkeysvalueof{/tikz/grid with coordinates/major step}}
\pgfmathsetmacro{\grd@yc}{\grd@ya + \pgfkeysvalueof{/tikz/grid with coordinates/major step}}
\foreach \x in {\grd@xa,\grd@xc,...,\grd@xb}
\node[anchor=north] at (\x,\grd@ya) {\pgfmathprintnumber{\x}};
\foreach \y in {\grd@ya,\grd@yc,...,\grd@yb}
\node[anchor=east] at (\grd@xa,\y) {\pgfmathprintnumber{\y}};
}
}
},
minor help lines/.style={
help lines, gray!20,
step=\pgfkeysvalueof{/tikz/grid with coordinates/minor step}
},
middle help lines/.style={
help lines, gray!40,
line width=\pgfkeysvalueof{/tikz/grid with coordinates/major line width},
step=\pgfkeysvalueof{/tikz/grid with coordinates/middle step}
},
major help lines/.style={
help lines, gray!80,
line width=\pgfkeysvalueof{/tikz/grid with coordinates/major line width},
step=\pgfkeysvalueof{/tikz/grid with coordinates/major step}
},
grid with coordinates/.cd,
minor step/.initial=.1,
middle step/.initial=.5,
middle line width/.initial=.5pt,
major step/.initial=1,
major line width/.initial=1pt,
}
\makeatother

\lstdefinelanguage{JavaScript}{
  keywords={break, case, catch, continue, debugger, default, delete, do, else, false, finally, for, function, if, in, instanceof, new, null, return, switch, this, throw, true, try, typeof, var, void, while, with},
  morecomment=[l]{//},
  morecomment=[s]{/*}{*/},
  morestring=[b]',
  morestring=[b]",
  ndkeywords={class, export, boolean, throw, implements, import, this},
  keywordstyle=\color{unibablueI},
  ndkeywordstyle=\color{unibagreenI},
  identifierstyle=\color{black},
  commentstyle=\color{unibagrayI}\ttfamily,
  stringstyle=\color{unibaredI}\ttfamily,
  sensitive=true
}


%% Fancy Quotes
\makeatletter
\tikzset{%
  fancy quotes/.style={
    text width=\fq@width pt,
    align=justify,
    inner sep=1em,
    anchor=north west,
    minimum width=\textwidth,
  },
  fancy quotes width/.initial={.8\textwidth},
  fancy quotes marks/.style={
    scale=8,
    text=white,
    inner sep=0pt,
  },
  fancy quotes opening/.style={
    fancy quotes marks,
  },
  fancy quotes closing/.style={
    fancy quotes marks,
  },
  fancy quotes background/.style={
    show background rectangle,
    inner frame xsep=0pt,
    background rectangle/.style={
      fill=unibagrayIV,
      rounded corners,
    },
  }
}

\newenvironment{fancyquotes}[1][]{%
\noindent
\tikzpicture[fancy quotes background]
\node[fancy quotes opening,anchor=north west] (fq@ul) at (0,0) {``};
\tikz@scan@one@point\pgfutil@firstofone(fq@ul.east)
\pgfmathsetmacro{\fq@width}{\textwidth - 2*\pgf@x}
\node[fancy quotes,#1] (fq@txt) at (fq@ul.north west) \bgroup}
{\egroup;
\node[overlay,fancy quotes closing,anchor=east] at (fq@txt.south east) {''};
\endtikzpicture}

\makeatother

\ifpresentation
\changemenucolor{gray}{bg}{named}{unibablueV}
\changemenucolor{gray}{br}{named}{unibablueI}
\changemenucolor{gray}{txt}{named}{unibablueI}
\fi

\newcommand{\cmark}{\ding{51}}%
\newcommand{\xmark}{\ding{55}}%
\newcommand*{\nodelabel}[1]{{\normalsize\bfseries\ttfamily #1}}%
\lstset{literate=
  {á}{{\'a}}1 {é}{{\'e}}1 {í}{{\'i}}1 {ó}{{\'o}}1 {ú}{{\'u}}1
  {Á}{{\'A}}1 {É}{{\'E}}1 {Í}{{\'I}}1 {Ó}{{\'O}}1 {Ú}{{\'U}}1
  {à}{{\`a}}1 {è}{{\`e}}1 {ì}{{\`i}}1 {ò}{{\`o}}1 {ù}{{\`u}}1
  {À}{{\`A}}1 {È}{{\`E}}1 {Ì}{{\`I}}1 {Ò}{{\`O}}1 {Ù}{{\`U}}1
  {ä}{{\"a}}1 {ë}{{\"e}}1 {ï}{{\"i}}1 {ö}{{\"o}}1 {ü}{{\"u}}1
  {Ä}{{\"A}}1 {Ë}{{\"E}}1 {Ï}{{\"I}}1 {Ö}{{\"O}}1 {Ü}{{\"U}}1
  {â}{{\^a}}1 {ê}{{\^e}}1 {î}{{\^i}}1 {ô}{{\^o}}1 {û}{{\^u}}1
  {Â}{{\^A}}1 {Ê}{{\^E}}1 {Î}{{\^I}}1 {Ô}{{\^O}}1 {Û}{{\^U}}1
  {ã}{{\~a}}1 {ẽ}{{\~e}}1 {ĩ}{{\~i}}1 {õ}{{\~o}}1 {ũ}{{\~u}}1
  {Ã}{{\~A}}1 {Ẽ}{{\~E}}1 {Ĩ}{{\~I}}1 {Õ}{{\~O}}1 {Ũ}{{\~U}}1
  {œ}{{\oe}}1 {Œ}{{\OE}}1 {æ}{{\ae}}1 {Æ}{{\AE}}1 {ß}{{\ss}}1
  {ű}{{\H{u}}}1 {Ű}{{\H{U}}}1 {ő}{{\H{o}}}1 {Ő}{{\H{O}}}1
  {ç}{{\c c}}1 {Ç}{{\c C}}1 {ø}{{\o}}1 {å}{{\r a}}1 {Å}{{\r A}}1
  {€}{{\euro}}1 {£}{{\pounds}}1 {«}{{\guillemotleft}}1
  {»}{{\guillemotright}}1 {ñ}{{\~n}}1 {Ñ}{{\~N}}1 {¿}{{?`}}1 {¡}{{!`}}1
}

\lstset{
  backgroundcolor=\color{white},   % choose the background color; you must add \usepackage{color} or \usepackage{xcolor}; should come as last argument
  basicstyle=\footnotesize,        % the size of the fonts that are used for the code
  breakatwhitespace=false,         % sets if automatic breaks should only happen at whitespace
  breaklines=true,                 % sets automatic line breaking
  captionpos=b,                    % sets the caption-position to bottom
  commentstyle=\color{unibagreenI},    % comment style
  deletekeywords={...},            % if you want to delete keywords from the given language
  escapeinside={\%*}{*)},          % if you want to add LaTeX within your code
  extendedchars=true,              % lets you use non-ASCII characters; for 8-bits encodings only, does not work with UTF-8
  keepspaces=true,                 % keeps spaces in text, useful for keeping indentation of code (possibly needs columns=flexible)
  keywordstyle=\bfseries\color{unibablueI},       % keyword style
  numbers=left,                    % where to put the line-numbers; possible values are (none, left, right)
  numbersep=5pt,                   % how far the line-numbers are from the code
  numberstyle=\tiny\color{unibagrayI}, % the style that is used for the line-numbers
  stringstyle=\color{unibaredI},     % string literal style
  tabsize=2,	                   % sets default tabsize to 2 spaces
  title=\lstname                   % show the filename of files included with \lstinputlisting; also try caption instead of title
}

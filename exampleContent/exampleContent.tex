\newacronym{ieee}{IEEE}{Institute of Electrical and Electronics Engineers}

\newglossaryentry{latex}
{
        name=latex,
        description={Is a mark up language specially suited for scientific documents}
}

\newglossaryentry{symb:Pi}
{
        name=\ensuremath{\pi},
        description={Geometrical value},
        type=symbolslist
}

\lstset{language=TeX}

\section{Einleitung}\label{sec:ein}
Einleitung nach \autoref{sec:ein}

\section{Hauptteil}\label{sec:haupt}
\subsection{Bilder und Grafiken}\label{subsec:grafiken}

\subsubsection{Bilder}\label{subsubsec:bilder}
Bilder befinden sich im Image-Ordner und lassen sich mit \lstinline|\image{Breite}{Datei im Image-Verzeichnis}{Beschriftung}{Label}| einbinden. \image{3cm}{logo.png}{Uni-Logo}{img:uni} Die Referenzierung erfolgt mittels \lstinline|\autoref{Label}|, also z.B. \autoref{img:uni}.

\subsubsection{Grafiken mit TikZ}
Grafiken im TikZ-Framework\footnote{\url{http://www.tn-home.de/TUGDD/Stuff/TikZ_final.pdf}} lassen sich mit dem Befehl \lstinline|\scaletikzimage{Datei im Image Verzeichnis}{Beschriftung}{Label}{Skalierungsfaktor}| einbinden. \scaletikzimage{tikz}{TikZ-Grafik}{img:tikz}{0.9}

\paragraph{Grafiken mit Moeptikz}

Grafiken mit Netzwerksymbolen können einfach mit moeptikz\footnote{\url{https://github.com/moepinet/moeptikz}} erstellt und mit \lstinline|\scaletikzimage{}{}{}| eingebunden werden:
\scaletikzimage{network}{Example Network}{img:topo}{0.9}

\paragraph{Grafiken mit Tikz-UML}

UML basierte Grafiken können einfach mit tikz-uml\footnote{\url{https://perso.ensta-paris.fr/~kielbasi/tikzuml/index.php}} erstellt werden und mit \lstinline|\scaletikzimage{}{}{}| eingebunden werden:
\scaletikzimage{sequence}{Example Sequence}{img:seq}{0.9}

\subsection{Tabellen}
Tabellen, siehe \autoref{lst:table}, lassen sich mit dem Environment \lstinline|longtable| definieren\footnote{\url{ftp://ftp.dante.de/pub/tex/macros/latex/required/tools/longtable.pdf}}.

\begin{lstlisting}[caption=Tabelle, language=TeX, label=lst:table]
\begin{longtable}[H h t b c]{Spaltendefinitionen} \\
Zelle 1 & Zelle 2 & ... & Zelle n \\
... \\
Zelle x & Zelle y & ... & Zelle z \\
\caption{Tabellenunterschrift}
\label{Label}
\end{longtable}
\end{lstlisting}

\begin{longtable}[H]{|p{0.2\textwidth}|p{0.2\textwidth}|p{0.2\textwidth}|}
\hline
Zelle 1 & Zelle 2 & Zelle n \\
\hline
Zelle x & Zelle y & Zelle z \\
\hline
\caption{Tabelle 1}
\label{tab:tab1}
\end{longtable}

\subsection{Code-Ausschnitte}

\subsubsection{Pseudo-Code}
Pseudo-Code Ausschnitte lassen sich mit \lstinline|\pseudo{Name des Algorithmus}{Label}{Datei im Code-Verzeichnis}| einbinden.
\pseudo{Mittelwert}{lst:mean}{code}

\subsubsection{Programmiersprachen}
Code Ausschnitte lassen sich einfach mit listings verwenden z.B. in der Umgebung \lstinline|\begin{lstlisting} ... \end{lstlisting}|, wie in \autoref{lst:python} gezeigt.

\begin{lstlisting}[language=Python, caption={Simple Python program}, label=lst:python]
def my_function():
  print("Hello from a function")
\end{lstlisting}


Mit \lstinline|\nocite*{}| lassen sich alle Einträge in der Bibliography ausgeben. Mit \lstinline|\cite[S. xx]{Key}| lassen sich Zitate einfügen. Z.B. \cite[S. 234]{Kurose12} \nocite*{}

\section{Abkürzungen und Glossar}
\subsection{Abkürzungen}
Abkürzungen können mit \lstinline|\newacronym{ieee}{IEEE}{Institute of Electrical and Electronics Engineers}| angegeben werden. Diese werden alphabetisch sortiert in ein Abkürzungsverzeichnis aufgenommen und im Text z.B. mit \lstinline|\gls{ieee}| referenziert, dies führt bei Verwendung zu \gls{ieee}.

\subsection{Glossar}
Glossareinträge können, wie in \autoref{lst:glossar} gezeigt, angelegt werden. Diese können im Text ebenfalls mit \lstinline|\gls{latex}| referenziert werden, z.B. \gls{latex}.
\begin{lstlisting}[caption=Glossareinträge, label=lst:glossar, language=TeX]
 \newglossaryentry{latex}{
        name=latex,
        description={Is a mark up language specially suited for scientific documents}}
\end{lstlisting}

\subsection{Symbolverzeichnis}
Symbole aus dem Symbolverzeichnis können ebenfalls mit \lstinline|\gls{symb:Pi}| genutzt werden und erzeugen dann \gls{symb:Pi}, wenn der Eintrag wie in \autoref{lst:symbol}
\begin{lstlisting}[caption=Einträge für das Symbolverzeichnis, label=lst:symbol, language=TeX]
\newglossaryentry{symb:Pi}{
        name=\ensuremath{\pi},
        description={Geometrical value},
        type=symbolslist}
\end{lstlisting}
